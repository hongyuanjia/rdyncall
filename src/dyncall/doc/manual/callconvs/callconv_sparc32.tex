%//////////////////////////////////////////////////////////////////////////////
%
% Copyright (c) 2012-2022 Daniel Adler <dadler@uni-goettingen.de>,
%                         Tassilo Philipp <tphilipp@potion-studios.com>
%
% Permission to use, copy, modify, and distribute this software for any
% purpose with or without fee is hereby granted, provided that the above
% copyright notice and this permission notice appear in all copies.
%
% THE SOFTWARE IS PROVIDED "AS IS" AND THE AUTHOR DISCLAIMS ALL WARRANTIES
% WITH REGARD TO THIS SOFTWARE INCLUDING ALL IMPLIED WARRANTIES OF
% MERCHANTABILITY AND FITNESS. IN NO EVENT SHALL THE AUTHOR BE LIABLE FOR
% ANY SPECIAL, DIRECT, INDIRECT, OR CONSEQUENTIAL DAMAGES OR ANY DAMAGES
% WHATSOEVER RESULTING FROM LOSS OF USE, DATA OR PROFITS, WHETHER IN AN
% ACTION OF CONTRACT, NEGLIGENCE OR OTHER TORTIOUS ACTION, ARISING OUT OF
% OR IN CONNECTION WITH THE USE OR PERFORMANCE OF THIS SOFTWARE.
%
%//////////////////////////////////////////////////////////////////////////////

\subsection{SPARC Calling Conventions}

\paragraph{Overview}

The SPARC family of processors is based on the SPARC instruction set architecture, which comes in basically three revisions,
V7, V8\cite{SPARCV8}\cite{SPARCSysV} and V9\cite{SPARCV9}\cite{SPARCV9SysV}. The former two are 32-bit whereas the latter refers to the 64-bit SPARC architecture (see next chapter).
SPARC uses big endian byte order.\\
The word size is defined to be 32 bits.

\paragraph{\product{dyncall} support}

\product{dyncall} fully supports the SPARC 32-bit instruction set (V7 and V8), for calls and callbacks.

\subsubsection{SPARC (32-bit) Calling Convention}

\paragraph{Register usage}

\begin{itemize}
\item 32 single floating point registers (f0-f31, usable as 8 quad precision q0,q4,q8,...,q28, 16 double precision d0,d2,d4,...,d30)
\item 32 32-bit integer/pointer registers out of a bigger (vendor/model dependent) number that are accessible at a time (8 are global ones (g*), whereas the remaining 24 form a register window with 8 input (i*), 8 output (o*) and 8 local (l*) ones)
\item calling a function shifts the register window, the old output registers become the new input registers (old local and input ones are not accessible anymore)
\end{itemize}

\begin{table}[h]
\begin{tabular*}{0.95\textwidth}{lll}
Name                          & Alias                          & Brief description\\
\hline
{\bf \%g0}                    & \%r0                           & Read-only, hardwired to 0 \\
{\bf \%g1-\%g7}               & \%r1-\%r7                      & Global \\
{\bf \%o0,\%o1 and \%i0,\%i1} & \%r8,\%r9 and \%r24,\%r25      & Output and input argument registers, return value \\
{\bf \%o2-\%o5 and \%i2-\%i5} & \%r10-\%r13 and \%r26-\%r29    & Output and input argument registers \\
{\bf \%o6 and \%i6}           & \%r14 and \%r30, \%sp and \%fp & Stack and frame pointer \\
{\bf \%o7 and \%i7}           & \%r15 and \%r31                & Return address (caller writes to o7, callee uses i7) \\
{\bf \%l0-\%l7}               & \%r16-\%r23                    & preserve \\
{\bf \%f0,\%f1}               &                                & Floating point return value \\
{\bf \%f2-\%f31}              &                                & scratch \\
\end{tabular*}
\caption{Register usage on sparc calling convention}
\end{table}

\paragraph{Parameter passing}
\begin{itemize}
\item stack grows down
\item stack parameter order: right-to-left
\item caller cleans up the stack
\item stack always aligned to 8 bytes
\item first 6 integers/pointers and floats are passed independently in registers using \%o0-\%o5
\item for every other argument the stack is used
\item all arguments \textless=\ 32 bit are passed as 32 bit values
\item 64 bit arguments are passed like two consecutive \textless=\ 32 bit values (which allows for an argument to be split between the stack and \%i5)
\item aggregates (struct, union) of any size, as well as quad precision values are passed indirectly as a pointer to a {\bf copy} of the aggregate (like: struct s2 = s; callee(\&s2);)
\item {\it non-trivial} C++ aggregates (as defined by the language) of any size, are passed indirectly via a pointer to a copy of the aggregate
\item minimum stack size is 64 bytes, b/c stack pointer must always point at enough space to store all \%i* and \%l* registers, used when running out of register windows
\item if needed, register spill area is adjacent to parameters
\end{itemize}

\paragraph{Return values}

\begin{itemize}
\item results are expected by caller to be returned in \%o0/\%o1 (after reg window restore, meaning callee writes to \%i0/\%i1) for integers
\item \%f0/\%f1 are used for floating point values
\item aggregates (struct, union) and quad precision values are returned in a space allocated by the caller, with a pointer to it passed
as an {\bf additional}, hidden {\bf stack} parameter (always at \%sp+64 for the caller, see below); that pointer is returned in \%o0
\end{itemize}

\paragraph{Stack layout}

% verified/amended: TP nov 2019 (see also doc/disas_examples/sparc.sparc.disas)
Stack directly after function prolog:\\

\begin{figure}[h]
\begin{tabular}{5|3|1 1}
                                   & \vdots                      &                                &                               \\
\hhline{~=~~}
local data (and padding)           & \hspace{4cm}                &                                & \mrrbrace{9}{caller's frame}  \\
\hhline{~-~~}
\mrlbrace{7}{parameter area}       & arg n-1                     & \mrrbrace{3}{stack parameters} &                               \\
                                   & \ldots                      &                                &                               \\
                                   & 7th word of arg data        &                                &                               \\
                                   & \%o5                        & \mrrbrace{3}{spill area}       &                               \\
                                   & \ldots                      &                                &                               \\
                                   & \%o0                        &                                &                               \\
                                   & struct/union return pointer &                                &                               \\
\hhline{~-~~}
register save area (\%i* and \%l*) &                             &                                &                               \\
\hhline{~=~~}
local data (and padding)           &                             &                                & \mrrbrace{3}{current frame}   \\
\hhline{~-~~}
parameter area                     &                             &                                &                               \\
\hhline{~-~~}
                                   & \vdots                      &                                &                               \\
\end{tabular}
\caption{Stack layout on sparc32 calling convention}
\end{figure}

