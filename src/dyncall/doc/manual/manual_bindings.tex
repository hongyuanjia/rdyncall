%//////////////////////////////////////////////////////////////////////////////
%
% Copyright (c) 2007,2009 Daniel Adler <dadler@uni-goettingen.de>, 
%                         Tassilo Philipp <tphilipp@potion-studios.com>
%
% Permission to use, copy, modify, and distribute this software for any
% purpose with or without fee is hereby granted, provided that the above
% copyright notice and this permission notice appear in all copies.
%
% THE SOFTWARE IS PROVIDED "AS IS" AND THE AUTHOR DISCLAIMS ALL WARRANTIES
% WITH REGARD TO THIS SOFTWARE INCLUDING ALL IMPLIED WARRANTIES OF
% MERCHANTABILITY AND FITNESS. IN NO EVENT SHALL THE AUTHOR BE LIABLE FOR
% ANY SPECIAL, DIRECT, INDIRECT, OR CONSEQUENTIAL DAMAGES OR ANY DAMAGES
% WHATSOEVER RESULTING FROM LOSS OF USE, DATA OR PROFITS, WHETHER IN AN
% ACTION OF CONTRACT, NEGLIGENCE OR OTHER TORTIOUS ACTION, ARISING OUT OF
% OR IN CONNECTION WITH THE USE OR PERFORMANCE OF THIS SOFTWARE.
%
%//////////////////////////////////////////////////////////////////////////////

\clearpage
\section{Bindings to programming languages}

Through binding of the \product{dyncall} library into a scripting environment,
the scripting language can gain system programming status to a certain degree.\\
The \product{dyncall} library provides bindings to Erlang\cite{Erlang}, Java\cite{Java},
Lua\cite{Lua}, Python\cite{Python}, R\cite{R}, Ruby\cite{Ruby}, Go\cite{Go} and the shell/command line.\\
However, please note that some of these bindings are work-in-progress and not
automatically tested, meaning it might require some additional work to make them
work.

\subsection{Common Architecture}

The binding interfaces of the \product{dyncall} library to various scripting
languages share a common set of functionality to invoke a function call.

\subsubsection{Dynamic loading of code}

The helper library \emph{dynload} which accompanies the \product{dyncall}
library provides an abstract interface to operating-system specific mechanisms
for loading and accessing executable code out of, but not limited to, shared
libraries.

\subsubsection{Functions}

All bindings are based on a common interface convention providing a common set
of the following 4 functions (exact spelling depending on the binding's scripting
environment):
\begin{description}
\item [load] - load a module of compiled code
\item [free] - unload a module of compiled code
\item [find] - find function pointer by symbolic names
\item [call] - invoke a function call
\end{description}

\pagebreak

\subsubsection{Signatures}

A signature is a character string that represents a function's arguments and
return value types. It is used in the scripting language bindings invoke
functions to perform automatic type-conversion of the languages' types to the
low-level C/C++ data types.
This is an essential part of mapping the more flexible and often abstract data
types provided in scripting languages to the strict machine-level data types
used by C-libraries.
The high-level C interface functions \capi{dcCallF()}, \capi{dcVCallF()},
\capi{dcArgF()} and \capi{dcVArgF()} of the \product{dyncall} library also make
use of this signature string format.\\
\\
The format of a \product{dyncall} signature string is as depicted below:


\paragraph{\product{dyncall} signature string format}

\begin{center}
\group{input parameter type signature character}* \sigchar{)} \group{return
type signature character} \\
\end{center}

The \group{input parameter type signature character} sequence left to the
\sigchar{)} is in left-to-right order of the corresponding C function
parameter type list.\\
The special \group{return type signature character} \sigchar{v} specifies
that the function does not return a value and corresponds to \capi{void}
functions in C.

\begin{table}[h]
\begin{center}
\begin{tabular*}{0.75\textwidth}{cl}
Signature character & C/C++ data type \\
\hline
\sigchar{v} & void \\
\sigchar{B} & \_Bool, bool \\
\sigchar{c} & char \\
\sigchar{C} & unsigned char \\
\sigchar{s} & short \\
\sigchar{S} & unsigned short \\
\sigchar{i} & int \\
\sigchar{I} & unsigned int \\
\sigchar{j} & long \\
\sigchar{J} & unsigned long \\
\sigchar{l} & long long, int64\_t \\
\sigchar{L} & unsigned long long, uint64\_t \\
\sigchar{f} & float \\
\sigchar{d} & double \\
\sigchar{p} & void* \\
\sigchar{Z} & const char* (pointing to C string) \\
\sigchar{A} & aggregate (struct, union) by-value \\
\end{tabular*}
\caption{Type signature encoding for function call data types}
\label{sigchar}
\end{center}
\end{table}

Please note that using a \sigchar{(} at the beginning of a signature string is possible,
although not required. The character doesn't have any meaning and will simply be
ignored. However, using it prevents annoying syntax highlighting problems with some code
editors.

\pagebreak

Calling convention modes can be switched using the signature string, as well. A
'\_' in the signature string is followed by a character specifying what
calling convention to use, as this affects how arguments are passed. This makes
only sense if there are multiple co-existing calling conventions on a single platform.
Usually, this is done at the beginning of the string, except in special cases, like
specifying where the varargs part of a variadic function begins.
The following signature characters exist:

\begin{table}[h]
\begin{center}
\begin{tabular*}{0.75\textwidth}{cl}
Signature character & Calling Convention \\
\hline
\sigchar{:} & platform's default calling convention \\
\sigchar{*} & platform's default C++/thiscall calling convention \\
\sigchar{e} & vararg function \\
\sigchar{.} & vararg function's variadic/ellipsis part (...), to be specified before first vararg \\
\sigchar{c} & only on x86: cdecl \\
\sigchar{s} & only on x86: stdcall \\
\sigchar{F} & only on x86: fastcall (MS) \\
\sigchar{f} & only on x86: fastcall (GNU) \\
\sigchar{+} & only on x86: thiscall (MS) \\
\sigchar{\#}& only on x86: thiscall (GNU) \\
\sigchar{A} & only on ARM: ARM mode \\
\sigchar{a} & only on ARM: THUMB mode \\
\sigchar{\$}& syscall \\
\end{tabular*}
\caption{Calling convention signature encoding}
\label{cconvsigchar}
\end{center}
\end{table}

\pagebreak

\paragraph{Examples of C function prototypes}

\begin{table}[h]
\begin{tabular*}{0.75\textwidth}{rll}
& C function prototype & dyncall signature \\
\hline
void      & f1();                                     & \sigstr{)v}\\
int       & f2(int, int);                             & \sigstr{ii)i}\\
long long & f3(void*);                                & \sigstr{p)L}\\
void      & f3(int**);                                & \sigstr{p)v}\\
double    & f4(int, bool, char, double, const char*); & \sigstr{iBcdZ)d}\\
void      & f5(short, long long, ...);                & \sigstr{\_esl\_.di)v} (for (promoted) varargs: double, int)\\
struct A  & f6(int, union B);                         & \sigstr{iA)A}\\
short     & Cls::f(unsigned char, ...);               & \sigstr{\_*p\_eC\_.i)s} (C++: this-ptr as 1st arg, int as vararg)\\
\end{tabular*}
\caption{Type signature examples of function prototypes}
\label{sigex}
\end{table}



\subsection{Erlang language bindings}

The OTP library application {\tt erldc} implements the Erlang language bindings.

\begin{table}[h]
\begin{center}
\begin{tabular*}{0.75\textwidth}{ll}
Signature character & accepted Erlang data types\\
\hline
\sigchar{v}              & no return type\\
\sigchar{B} & atoms 'true' and 'false' converted to bool\\
\sigchar{c}, \sigchar{C} & integer cast to (unsigned) char\\
\sigchar{s}, \sigchar{S} & integer cast to (unsigned) short\\
\sigchar{i}, \sigchar{I} & integer cast to (unsigned) int\\
\sigchar{j}, \sigchar{J} & integer cast to (unsigned) long\\
\sigchar{l}, \sigchar{L} & integer cast to (unsigned) long long\\
\sigchar{f}              & decimal cast to float\\
\sigchar{d}              & decimal cast to double\\
\sigchar{p}              & binary (previously returned from call\_ptr or callf) cast to void*\\
\sigchar{Z}              & string cast to void*\\
\end{tabular*}
\caption{Type signature encoding for Erlang bindings}
\label{Erlangsigchar}
\end{center}
\end{table}

\pagebreak

\subsection{Go language bindings}

A Go binding is provided through the {\tt godc} package. Since Go's type system is basically a superset of C's, the type mapping from Go to C is straightforward.

\begin{table}[h]
\begin{center}
\begin{tabular*}{0.75\textwidth}{ll}
Signature character & accepted Go data types\\
\hline
\sigchar{v}              & no return type\\
\sigchar{B} & bool\\
\sigchar{c}, \sigchar{C} & int8, uint8\\
\sigchar{s}, \sigchar{S} & int16, uint16\\
\sigchar{i}, \sigchar{I} & int, uint\\
\sigchar{j}, \sigchar{J} & int32, uint32\\
\sigchar{l}, \sigchar{L} & int64, uint64\\
\sigchar{f}              & float32\\
\sigchar{d}              & float64\\
\sigchar{p}, \sigchar{Z} & uintptr, unsafe.Pointer\\
\end{tabular*}
\caption{Type signature encoding for Go bindings}
\label{Gosigchar}
\end{center}
\end{table}

Note that passing a Go-{\tt string} directly to a C-function expecting a pointer is not directly possible. However, the binding comes with
two helper functions, {\tt AllocCString(value string) unsafe.Pointer} and {\tt FreeCString(value unsafe.Pointer)} to help with converting
a {\tt string} to an {\tt unsafe.Pointer} which then can be passed to {\tt ArgPointer(value unsafe.Pointer)}. Once you are done with this
temporary string, free it using {\tt FreeCString(value unsafe.Pointer)}.


\subsection{Python language bindings}

The python module {\tt pydc} implements the Python language bindings,
namely {\tt load}, {\tt find}, {\tt free}, {\tt call}, {\tt new\_callback}, {\tt free\_callback}.

\begin{table}[h]
\begin{tabular*}{1.0\textwidth}{lll}
Signature character & accepted Python 2 types & accepted Python 3 types \\
\hline
\sigchar{v}              & no return type                                     & no return type                                \\
\sigchar{B}              & bool                                               & bool                                          \\
\sigchar{c}, \sigchar{C} & int, string (with single char)                     & int, string (with single char)                \\
\sigchar{s}, \sigchar{S} & int                                                & int                                           \\
\sigchar{i}, \sigchar{I} & int                                                & int                                           \\
\sigchar{j}, \sigchar{J} & int                                                & int                                           \\
\sigchar{l}, \sigchar{L} & int, long                                          & int                                           \\
\sigchar{f}              & float                                              & float                                         \\
\sigchar{d}              & float                                              & float                                         \\
\sigchar{p}              & bytearray, int, long, None, (PyCObject, PyCapsule) & bytearray, int, None, (PyCObject, PyCapsule)  \\
\sigchar{Z}              & string, unicode, bytearray                         & string, bytes, bytearray                      \\
\end{tabular*}
\caption{Type signature encoding for Python bindings}
\label{Pysigchar}
\end{table}

This is a very brief description that omits many details. For more, refer to the README.txt file of the binding.

\pagebreak

\subsection{R language bindings}

The R package {\tt rdyncall} implements the R langugae bindings providing the function
{\tt .dyncall() }.

\begin{table}[h]
\begin{center}
\begin{tabular*}{0.75\textwidth}{ll}
Signature character & accepted R data types\\
\hline
\sigchar{v} & no return type\\
\sigchar{B} & coerced to logical vector, first item\\
\sigchar{c} & coerced to integer vector, first item truncated char\\
\sigchar{C} & coerced to integer vector, first item truncated to unsigned char\\
\sigchar{s} & coerced to integer vector, first item truncated to short\\
\sigchar{S} & coerced to integer vector, first item truncated to unsigned short\\
\sigchar{i} & coerced to integer vector, first item\\
\sigchar{I} & coerced to integer vector, first item casted to unsigned int\\
\sigchar{j} & coerced to integer vector, first item\\
\sigchar{J} & coerced to integer vector, first item casted to unsigned long\\
\sigchar{l} & coerced to numeric, first item casted to long long\\
\sigchar{L} & coerced to numeric, first item casted to unsigned long long\\
\sigchar{f} & coerced to numeric, first item casted to float\\
\sigchar{d} & coerced to numeric, first item\\
\sigchar{p} & external pointer or coerced to string vector, first item\\
\sigchar{Z} & coerced to string vector, first item\\
\end{tabular*}
\caption{Type signature encoding for R bindings}
\label{Rsigchar}
\end{center}
\end{table}

Some notes on the R Binding:
\begin{itemize}
\item Unsigned 32-bit integers are represented as signed integers in R.
\item 64-bit integer types do not exist in R, therefore we use double floats
to represent 64-bit integers (using only the 52-bit mantissa part).
\end{itemize}

\pagebreak

\subsection{Ruby language bindings}

The Ruby gem {\tt rbdc} implements the Ruby language bindings.

\begin{table}[h]
\begin{center}
\begin{tabular*}{0.75\textwidth}{ll}
Signature character & accepted Ruby data types\\
\hline
\sigchar{v}              & no return type\\
\sigchar{B} & TrueClass, FalseClass, NilClass, Fixnum casted to bool\\
\sigchar{c}, \sigchar{C} & Fixnum cast to (unsigned) char\\
\sigchar{s}, \sigchar{S} & Fixnum cast to (unsigned) short\\
\sigchar{i}, \sigchar{I} & Fixnum cast to (unsigned) int\\
\sigchar{j}, \sigchar{J} & Fixnum cast to (unsigned) long\\
\sigchar{l}, \sigchar{L} & Fixnum cast to (unsigned) long long\\
\sigchar{f}              & Float cast to float\\
\sigchar{d}              & Float cast to double\\
\sigchar{p}, \sigchar{Z} & String cast to void*\\
\end{tabular*}
\caption{Type signature encoding for Ruby bindings}
\label{Rubysigchar}
\end{center}
\end{table}

